<section>
    <title>Section 1.6 Transformation of Functions</title>

    <p>
        Often when given a problem, we try to model the scenario using mathematics in the form of words, tables, graphs, and equations in order to explain or solve it.
        When building models, it is often helpful to build from existing formulas or models.
        Knowing the basic graphs of your tool-kit functions can help you solve problems by being able to model new behavior by adapting something you already know.
        Unfortunately, the models and existing formulas we know are not always the same as the ones presented in the problems we face.
    </p>

    <p>
        Fortunately, there are systematic ways to shift, stretch, compress, flip and combine functions to help them become better models for the problems we are trying to solve.
        We can transform what we already know into what we need, hence the name, “Transformation of functions.” When we have a story problem, formula, graph, or table, we can then transform that function in a variety of ways to form new functions.
    </p>


    <subsection>
        <title>Shifts</title>

        <example>
            <statement>
                <p>
                    To regulate temperature in a green building, air flow vents near the roof open and close throughout the day to allow warm air to escape.
                    The graph below shows the open vents <m>V</m> (in square feet) throughout the day, <m>t</m> in hours after midnight.
                    During the summer, the facilities staff decides to try to better regulate temperature by increasing the amount of open vents by 20 square feet throughout the day.
                    Sketch a graph of this new function.
                </p>

                <image source="C1S6Image1.png">
                </image>

                <p>
                    We can sketch a graph of this new function by adding 20 to each of the output values of the original function.
                    This will have the effect of shifting the graph up.
                </p>

                <image source="C1S6Image2.png">
                </image>

                <p>
                    Notice that in the second graph, for each input value, the output value has increased by twenty, so if we call the new function <m>S(t)</m>, we could write <m>S(t)=V(t)+20</m>.
                </p>

                <p>
                    Note that this notation tells us that for any value of <m>t</m>, <m>S(t)</m> can be found by evaluating the <m>V</m> function at the same input, then adding twenty to the result.
                    This defines <m>S</m> as a transformation of the function <m>V</m>, in this case a vertical shift up 20 units.
                </p>

                <p>
                    Notice that with a vertical shift the input values stay the same and only the output values change.
                </p>
            </statement>
        </example>

        <definition>
            <title>Vertical Shift</title>

            <statement>
                <p>
                    Given a function <m>f(x)</m>, if we define a new function <m>g(x)</m> as <m>g(x)=f(x)+k</m>, where <m>k</m> is a constant, then <m>g(x)</m> is a vertical shift of the function <m>f(x)</m>, where all the output values have been increased by <m>k</m>.
                    If <m>k</m> is positive, then the graph will shift up by <m>k</m> units.
                    If <m>k</m> is negative, then the graph will shift down by <m>k</m> units.
                </p>
            </statement>
        </definition>

        <example>
            <statement>
                <p>
                    A function <m>f(x)</m> is given as a table below.
                    Create a table for the function <m>g(x)=f(x)-3</m>.
                </p>

                <tabular>
                    <col halign="center"> </col> <col halign="center"> </col> <col halign="center"> </col> <col halign="center"> </col> <col halign="center"> </col>
                    <row>
                        <cell> <m>x</m> </cell>
                        <cell>2</cell>
                        <cell>4</cell>
                        <cell>6</cell>
                        <cell>8</cell>
                    </row>

                    <row>
                        <cell> <m>f(x)</m> </cell>
                        <cell>1</cell>
                        <cell>3</cell>
                        <cell>7</cell>
                        <cell>11</cell>
                    </row>
                </tabular>

                <p>
                    The formula <m>g(x)=f(x)-3</m> tells us that we can find the output values of the <m>g</m> function by subtracting 3 from the output values of the <m>f</m> function.
                    For example, <m>f(2)=1</m> is found from the given table.
                    <m>g(x)=f(x)-3</m> is our given transformation, so <m>g(2)=f(2)-3=1-3=-2</m>.
                    Subtracting 3 from each <m>f(x)</m> value, we can complete a table of values for <m>g(x)</m>,
                </p>

                <tabular>
                    <col halign="center"> </col> <col halign="center"> </col> <col halign="center"> </col> <col halign="center"> </col> <col halign="center"> </col>
                    <row>
                        <cell> <m>x</m> </cell>
                        <cell>2</cell>
                        <cell>4</cell>
                        <cell>6</cell>
                        <cell>8</cell>
                    </row>

                    <row>
                        <cell> <m>g(x)</m> </cell>
                        <cell>-2</cell>
                        <cell>0</cell>
                        <cell>4</cell>
                        <cell>8</cell>
                    </row>
                </tabular>
            </statement>
        </example>

        <p>
            As with the earlier vertical shift, notice the input values stay the same and only the output values change.
        </p>

        <exercise>
            <statement>
                <p>
                    The function <m>a(t)=-4.9t^{2}+30t</m> gives the height <m>h</m> of a ball (in meters) thrown upwards from the ground after <m>t</m> seconds.
                    Suppose the ball was instead thrown from the top of a 10-meter building.
                    Relate this new height function <m>b(t)</m> to <m>a(t)</m>, then find a formula for <m>b(t)</m>.
                </p>
            </statement>
        </exercise>

        <p>
            The vertical shift is a change to the output, or outside, of the function.
            We will now look at how changes to input, on the inside of the function, change its graph and meaning.
        </p>

        <example>
            <statement>
                <p>
                    Returning to our building air flow example from the beginning of the section, suppose that in Fall, the facilities staff decides that the original venting plan starts too late, and they want to move the entire venting program to start two hours earlier.
                    Sketch a graph of the new function.
                </p>

                <image source="C1S6Image1.png">
                </image>
                $V(t)=$ the original venting plan
                <image source="C1S6Image3.png">
                </image>
                $F(t)=$ starting 2 hours sooner
                <p>
                    In the new graph, which we can call <m>F(t)</m>, at each time, the air flow is the same as the original function <m>V(t)</m> was two hours later.
                    For example, in the original function <m>V</m>, the air flow starts to change at 8 am, while for the function <m>F(t)</m> the air flow starts to change at 6 am.
                    The comparable function values are <m>V(8)=F(6)</m>.
                </p>

                <p>
                    Notice also that the vents first opened to 220 sq.
                    ft.
                    at 10 am under the original plan, while under the new plan the vents reach 220 sq.
                    ft.
                    at 8 am, so <m>V(10)=F(8)</m>.
                </p>

                <p>
                    In both cases we see that since <m>F(t)</m> starts 2 hours sooner, the same output values are reached when, <m>F(t)=V(t+2)</m>.
                </p>

                <p>
                    Note that <m>V(t+2)</m> had the effect of shifting the graph to the left.
                </p>
            </statement>
        </example>

        <p>
            Horizontal changes or “inside changes” affect the domain of a function (the input) instead of the range and often seem counterintuitive.
            The new function <m>F(t)</m> uses the same outputs as <m>V(t)</m> but matches those outputs to inputs two hours earlier than those of <m>V(t)</m>.
            Said another way, we must add 2 hours to the input of <m>V</m> to find the corresponding output for <m>F</m>: <m>F(t)=V(t+2)</m>.
        </p>

        <definition>
            <title>Horizontal Shift</title>

            <statement>
                <p>
                    Given a function <m>f(x)</m>, if we define a new function <m>g(x)</m> as <m>g(x)=f(x+k)</m>, where <m>k</m> is a constant, then <m>g(x)</m> is a horizontal shift of the function <m>f(x)</m>.
                    If <m>k</m> is positive, then the graph will shift left by <m>k</m> units.
                    If <m>k</m> is negative, then the graph will shift right by <m>k</m> units.
                </p>
            </statement>
        </definition>

        <example>
            <statement>
                <p>
                    A function <m>f(x)</m> is given as a table below.
                    Create a table for the function <m>g(x)=f(x-3)</m>.
                </p>

                <tabular>
                    <col halign="center"> </col> <col halign="center"> </col> <col halign="center"> </col> <col halign="center"> </col> <col halign="center"> </col>
                    <row>
                        <cell> <m>x</m> </cell>
                        <cell>2</cell>
                        <cell>4</cell>
                        <cell>6</cell>
                        <cell>8</cell>
                    </row>

                    <row>
                        <cell> <m>f(x)</m> </cell>
                        <cell>1</cell>
                        <cell>3</cell>
                        <cell>7</cell>
                        <cell>11</cell>
                    </row>
                </tabular>

                <p>
                    The formula <m>g(x)=f(x-3)</m> tells us that the output values of <m>g</m> are the same as the output value of <m>f</m> with an input value three smaller.
                    For example, we know that <m>f(2)=1</m>.
                    To get the same output from the <m>g</m> function, we will need an input value that is 3 larger: We input a value that is three larger for <m>g(x)</m> because the function takes three away before evaluating the function <m>f</m>.
                </p>

                <p>
                    <m>g(5)=f(5-3)=f(2)=1</m>
                </p>

                <tabular>
                    <col halign="center"> </col> <col halign="center"> </col> <col halign="center"> </col> <col halign="center"> </col> <col halign="center"> </col>
                    <row>
                        <cell> <m>x</m> </cell>
                        <cell>5</cell>
                        <cell>7</cell>
                        <cell>9</cell>
                        <cell>11</cell>
                    </row>

                    <row>
                        <cell> <m>g(x)</m> </cell>
                        <cell>1</cell>
                        <cell>3</cell>
                        <cell>7</cell>
                        <cell>11</cell>
                    </row>
                </tabular>

                <p>
                    The result is that the function <m>g(x)</m> has been shifted to the right by 3.
                    Notice the output values for <m>g(x)</m> remain the same as the output values for <m>f(x)</m> in the chart, but the corresponding input values, <m>x</m>, have shifted to the right by 3: 2 shifted to 5, 4 shifted to 7, 6 shifted to 9 and 8 shifted to 11.
                </p>
            </statement>
        </example>

        <example>
            <statement>
                <p>
                    The graph shown is a transformation of the toolkit function <m>f(x)=x^{2}</m>.
                    Relate this new function <m>g(x)</m> to <m>f(x)</m>, and then find a formula for <m>g(x)</m>.
                </p>

                <image source="C1S6Image4.png">
                </image>

                <p>
                    Notice that the graph looks almost identical in shape to the <m>f(x)=x^{2}</m> function, but the <m>x</m> values are shifted to the right two units.
                    The vertex used to be at <m>(0, 0)</m> but now the vertex is at <m>(2, 0)</m>.
                    The graph is the basic quadratic function shifted two to the right, so <m>g(x)=f(x-2)</m>.
                </p>

                <p>
                    Notice how we must input the value <m>x = 2</m>, to get the output value <m>y = 0</m>; the <m>x</m> values must be two units larger, because of the shift to the right by 2 units.
                </p>

                <p>
                    We can then use the definition of the <m>f(x)</m> function to write a formula for <m>g(x)</m> by evaluating <m>f(x-2)</m>: Since <m>f(x)=x^{2}</m> and <m>g(x)=f(x-2)</m>.
                    Thus, <m>g(x)=f(x-2)=(x-2)^{2}</m>.
                </p>

                <p>
                    If you find yourself having trouble determining whether the shift is +2 or -2, it might help to consider a single point on the graph.
                    For a quadratic, looking at the bottom-most point is convenient.
                    In the original function, <m>f(0)=0</m>.
                    In our shifted function, <m>g(2)=0</m>.
                    To obtain the output value of 0 from the <m>f</m> function, we need to decide whether a +2 or -2 will work to satisfy <m>g(2)=f(2-2)=f(0)=0</m>.
                    For this to work, we will need to subtract 2 from our input values.
                </p>
            </statement>
        </example>

        <p>
            When thinking about horizontal and vertical shifts, it is good to keep in mind that vertical shifts are affecting the output values of the function, while horizontal shifts are affecting the input values of the function.
        </p>

        <example>
            <statement>
                <p>
                    The function <m>G(m)</m> gives the number of gallons of gas required to drive <m>m</m> miles.
                    Interpret <m>G(m)+10</m> and <m>G(m+10)</m>.
                </p>

                <p>
                    <m>G(m)+10</m> is adding 10 to the output, gallons. This is 10 gallons of gas more than is required to drive <m>m</m> miles. So, this is the gas required to drive <m>m</m> miles, plus another 10 gallons of gas.
                </p>

                <p>
                    <m>G(m+10)</m> is adding 10 to the input, miles. This is the number of gallons of gas required to drive 10 miles more than <m>m</m> miles.
                </p>
            </statement>
        </example>

        <exercise>
            <statement>
                <p>
                    Given the function <m>f(x)=\sqrt{x}</m> graph the original function <m>f(x)</m> and the transformation <m>g(x)=f(x+2)</m>.
                    <ol>
                        <li>
                            <p>
                                Is this a horizontal or a vertical change?
                            </p>
                        </li>

                        <li>
                            <p>
                                Which way is the graph shifted and by how many units?
                            </p>
                        </li>

                        <li>
                            <p>
                                Graph <m>f(x)</m> and <m>g(x)</m> on the same axes.
                            </p>
                        </li>
                    </ol>
                </p>
            </statement>
        </exercise>

        <p>
            Now that we have two transformations, we can combine them together.
        </p>

        <p>
            Remember:
        </p>

        <ul>
            <li>
                <p>
                    Vertical Shifts are outside changes that affect the output (vertical) axis values shifting the transformed function up or down.
                </p>
            </li>

            <li>
                <p>
                    Horizontal Shifts are inside changes that affect the input (horizontal) axis values shifting the transformed function left or right.
                </p>
            </li>
        </ul>

        <example>
            <statement>
                <p>
                    Given <m>f(x)=|x|</m>, sketch a graph of <m>h(x)=f(x+1)-3</m>.
                </p>

                <p>
                    The function <m>f</m> is our toolkit absolute value function.
                    We know that this graph has a V-shape, with the point at the origin.
                    The graph of <m>h</m> has transformed <m>f</m> in two ways: <m>f(x+1)</m> is a change on the inside of the function, giving a horizontal shift left by 1, then the subtraction by 3 in <m>f(x+1)-3</m> is a change to the outside of the function, giving a vertical shift down by 3.
                    Transforming the graph gives:
                </p>

                <image source="C1S6Image5.png">
                </image>

                <p>
                    We could also find a formula for this transformation by evaluating the expression for <m>h(x)</m>: <m>h(x)=f(x+1)-3=|x+1|-3</m>.
                </p>
            </statement>
        </example>

        <example>
            <statement>
                <p>
                    Write a formula for the graph shown, which is a transformation of the toolkit square root function.
                </p>

                <image source="C1S6Image6.png">
                </image>

                <p>
                    The graph of the toolkit function starts at the origin, so this graph has been shifted 1 to the right, and up 2.
                    In function notation, we could write that as <m>h(x)=f(x-1)+2</m>.
                    Using the formula for the square root function we can write <m>h(x)=\sqrt{x-1}+2</m>.
                </p>

                <p>
                    Note that this transformation has changed the domain and range of the function.
                    This new graph has domain <m>[1,\infty)</m> and range <m>[2,\infty)</m>.
                </p>
            </statement>
        </example>
    </subsection>


    <subsection>
        <title>Reflections</title>

        <p>
            Another transformation that can be applied to a function is a reflection over the horizontal or vertical axis.
        </p>

        <example>
            <statement>
                <p>
                    Reflect the graph of <m>s(t)=\sqrt{t}</m> both vertically and horizontally.
                </p>

                <p>
                    Reflecting the graph vertically, each output value will be reflected over the horizontal t axis:
                </p>

                <image source="C1S6Image7.png">
                </image>
                Graph of $s(t)=\sqrt{t}$
                <image source="C1S6Image8.png">
                </image>
                Graph of $s(t)$ reflected vertically.
                <p>
                    Since each output value is the opposite of the original output value, we can write <m>V(t)=-s(t)=-\sqrt{t}</m>.
                </p>

                <p>
                    Notice this is an outside change or vertical change that affects the output <m>s(t)</m> values, so the negative sign belongs outside of the function.
                    Reflecting horizontally, each input value will be reflected over the vertical axis.
                </p>

                <image source="C1S6Image9.png">
                </image>
                Graph of $s(t)$ reflected horizontally.
                <p>
                    Since each input value is the opposite of the original input value, we can write <m>H(t)=s(-t)=\sqrt{-t}</m>.
                </p>

                <p>
                    Notice this is an inside change or horizontal change that affects the input values, so the negative sign is on the inside of the function.
                </p>

                <p>
                    Note that these transformations can affect the domain and range of the functions.
                    While the original square root function has domain <m>[0,\infty)</m> and range <m>[0,\infty)</m>, the vertical reflection gives the <m>V(t)</m> function the range <m>(-\infty,0]</m>, and the horizontal reflection gives the <m>H(t)</m> function the domain <m>(-\infty,0]</m>.
                </p>
            </statement>
        </example>

        <definition>
            <title>Reflections</title>

            <statement>
                <p>
                    Given a function <m>f(x)</m>, if we define a new function <m>g(x)</m> as <m>g(x)=-f(x)</m>, then <m>g(x)</m> is a vertical reflection of the function <m>f(x)</m>, sometimes called a reflection about the horizontal axis.
                </p>

                <p>
                    If we define a new function <m>g(x)</m> as <m>g(x)=f(-x)</m>, then <m>g(x)</m> is a horizontal reflection of the function <m>f(x)</m>, sometimes called a reflection about the vertical axis.
                </p>
            </statement>
        </definition>

        <example>
            <statement>
                <p>
                    A function <m>f(x)</m> is given as a table below.
                    Create a table for the function <m>g(x)=-f(x)</m> and <m>h(x)=f(-x)</m>.
                </p>

                <tabular>
                    <col halign="center"> </col> <col halign="center"> </col> <col halign="center"> </col> <col halign="center"> </col> <col halign="center"> </col>
                    <row>
                        <cell> <m>x</m> </cell>
                        <cell>2</cell>
                        <cell>4</cell>
                        <cell>6</cell>
                        <cell>8</cell>
                    </row>

                    <row>
                        <cell> <m>f(x)</m> </cell>
                        <cell>1</cell>
                        <cell>3</cell>
                        <cell>7</cell>
                        <cell>11</cell>
                    </row>
                </tabular>

                <p>
                    For <m>g(x)</m>, this is a vertical reflection, so the <m>x</m> values stay the same and each output value will be the opposite of the original output value:
                </p>

                <tabular>
                    <col halign="center"> </col> <col halign="center"> </col> <col halign="center"> </col> <col halign="center"> </col> <col halign="center"> </col>
                    <row>
                        <cell> <m>x</m> </cell>
                        <cell>2</cell>
                        <cell>4</cell>
                        <cell>6</cell>
                        <cell>8</cell>
                    </row>

                    <row>
                        <cell> <m>g(x)</m> </cell>
                        <cell>-1</cell>
                        <cell>-3</cell>
                        <cell>-7</cell>
                        <cell>-11</cell>
                    </row>
                </tabular>

                <p>
                    For <m>h(x)</m>, this is a horizontal reflection, and each input value will be the opposite of the original input value and the <m>h(x)</m> values stay the same as the <m>f(x)</m> values:
                </p>

                <tabular>
                    <col halign="center"> </col> <col halign="center"> </col> <col halign="center"> </col> <col halign="center"> </col> <col halign="center"> </col>
                    <row>
                        <cell> <m>x</m> </cell>
                        <cell>-2</cell>
                        <cell>-4</cell>
                        <cell>-6</cell>
                        <cell>-8</cell>
                    </row>

                    <row>
                        <cell> <m>h(x)</m> </cell>
                        <cell>1</cell>
                        <cell>3</cell>
                        <cell>7</cell>
                        <cell>11</cell>
                    </row>
                </tabular>
            </statement>
        </example>

        <example>
            <statement>
                <p>
                    A common model for learning has an equation similar to <m>k(t)=-2^{-t}+1</m>, where <m>k(t)</m> is the percentage of mastery that can be achieved after <m>t</m> practice sessions.
                    This is a transformation of the function <m>f(t)=2^{t}</m> shown here.
                </p>

                <image source="C1S6Image10.png">
                </image>

                <p>
                    Sketch a graph of <m>k(t)</m>.
                    This equation combines three transformations into one equation.
                    <ul>
                        <li>
                            <p>
                                A horizontal reflection: <m>f(-t)=2^{-t}</m>
                            </p>
                        </li>

                        <li>
                            <p>
                                A vertical reflection: <m>-f(-t)=-2^{-t}</m>
                            </p>
                        </li>

                        <li>
                            <p>
                                A vertical shift up 1: <m>-f(-t)+1=-2^{-t}+1</m>
                            </p>
                        </li>
                    </ul>
                </p>

                <p>
                    We can sketch a graph by applying these transformations one at a time to the original function:
                </p>

                <image source="C1S6Image11.png">
                </image>
                The original graph of $f(t)=2^{t}$.
                <image source="C1S6Image12.png">
                </image>
                Horizontally Reflected: $f(-t)$.
                <image source="C1S6Image13.png">
                </image>
                Then vertically reflected: $-f(-t)$.
                <image source="C1S6Image14.png">
                </image>
                Then shifted vertically: $-f(-t)+1$.
                <p>
                    Note: As a model for learning, this function would be limited to a domain of <m>t\geq 0</m>, with corresponding range <m>[0,1)</m>.
                </p>
            </statement>
        </example>

        <exercise>
            <statement>
                <p>
                    Given the toolkit function <m>f(x)=x^{2}</m>, graph <m>g(x) = -f(x)</m> and <m>h(x) = f(-x)</m>.
                    Do you notice anything surprising?
                </p>
            </statement>
        </exercise>

        <p>
            Some functions exhibit symmetry, in which reflections result in the original graph.
            For example, reflecting the toolkit functions <m>f(x)=x^{2}</m> or <m>f(x)=|x|</m> about the <m>y</m>-axis will result in the original graph.
            We call these types of graphs symmetric about the <m>y</m>-axis.
        </p>

        <p>
            Likewise, if the graphs of <m>f(x)=x^{3}</m> or <m>f(x)=\dfrac{1}{x}</m> were reflected over both axes, the result would be the original graph:
        </p>

        <image source="C1S6Image15.png">
        </image>
        $f(x)=x^{3}$
        <image source="C1S6Image16.png">
        </image>
        $f(-x)=(-x)^{3}=-x^{3}$.
        <image source="C1S6Image17.png">
        </image>
        $-f(-x)=-(-x^{3})=x^{3}$
        <p>
            We call these graphs symmetric about the origin.
        </p>
    </subsection>


    <subsection>
        <title>Stretches and Compressions</title>

        <p>
            With shifts, we saw the effect of adding or subtracting to the inputs or outputs of a function.
            We now explore the effects of multiplying the inputs or outputs.
        </p>

        <p>
            Remember, we can transform the inside (input values) of a function, or we can transform the outside (output values) of a function.
            Each change has a specific effect that can be seen graphically.
        </p>

        <example>
            <statement>
                <p>
                    A function <m>P(t)</m> models the growth of a population of fruit flies.
                    The growth is shown in the graph.
                </p>

                <image source="C1S6Image18.png">
                </image>

                <p>
                    A scientist is comparing this to another population, <m>Q</m>, that grows the same way, but starts twice as large.
                    Sketch a graph of this population.
                </p>

                <p>
                    Since the population is always twice as large, the new population’s output values are always twice the original function output values.
                    Graphically, this would look like the second graph shown.
                </p>

                <p>
                    Symbolically, <m>Q(t)=2P(t)</m>.
                </p>

                <image source="C1S6Image19.png">
                </image>

                <p>
                    This means that for any input <m>t</m>, the value of the <m>Q</m> function is twice the value of the <m>P</m> function.
                    Notice the effect on the graph is a vertical stretching of the graph, where every point doubles its distance from the horizontal axis.
                    The input values, <m>t</m>, stay the same while the output values are twice as large as before.
                </p>
            </statement>
        </example>

        <definition>
            <title>Vertical Stretch and Compression</title>

            <statement>
                <p>
                    Given a function <m>f(x)</m>, if we define a new function <m>g(x)</m> as <m>g(x)=kf(x)</m>, where <m>k</m> is a constant then <m>g(x)</m> is a vertical stretch or compression of the function <m>f(x)</m>.
                </p>

                <p>
                    If <m>k > 1</m>, then the graph will be stretched vertically.
                    If <m>0&#x3C; k &#x3C; 1</m>, then the graph will be compressed vertically.
                    If <m>k &#x3C; 0</m>, then there will be combination of a vertical stretch or compression with a vertical reflection.
                </p>
            </statement>
        </definition>

        <example>
            <statement>
                <p>
                    A function <m>f(x)</m> is given as a table below.
                </p>

                <tabular>
                    <col halign="center"> </col> <col halign="center"> </col> <col halign="center"> </col> <col halign="center"> </col> <col halign="center"> </col>
                    <row>
                        <cell> <m>x</m> </cell>
                        <cell>2</cell>
                        <cell>4</cell>
                        <cell>6</cell>
                        <cell>8</cell>
                    </row>

                    <row>
                        <cell> <m>f(x)</m> </cell>
                        <cell>1</cell>
                        <cell>3</cell>
                        <cell>7</cell>
                        <cell>11</cell>
                    </row>
                </tabular>

                <p>
                    Create a table for the function <m>g(x)=\dfrac{1}{2}f(x)</m>.
                    The formula <m>g(x)=\dfrac{1}{2}f(x)</m> tells us that the output values of <m>g</m> are half of the output values of <m>f</m> with the same inputs.
                    For example, we know that <m>f(4)=3</m>.
                    Then <m>g(4)=\dfrac{1}{2}f(4)=\dfrac{1}{2}(3)=\dfrac{3}{2}</m>.
                </p>

                <tabular>
                    <col halign="center"> </col> <col halign="center"> </col> <col halign="center"> </col> <col halign="center"> </col> <col halign="center"> </col>
                    <row>
                        <cell> <m>x</m> </cell>
                        <cell>2</cell>
                        <cell>4</cell>
                        <cell>6</cell>
                        <cell>8</cell>
                    </row>

                    <row>
                        <cell> <m>g(x)</m> </cell>
                        <cell> <m>\dfrac{1}{2}</m> </cell>
                        <cell> <m>\dfrac{3}{2}</m> </cell>
                        <cell> <m>\dfrac{7}{2}</m> </cell>
                        <cell> <m>\dfrac{11}{2}</m> </cell>
                    </row>
                </tabular>

                <p>
                    The result is that the function <m>g(x)</m> has been compressed vertically by <m>\dfrac{1}{2}</m>.
                    Each output value has been cut in half, so the graph would now be half the original height.
                </p>
            </statement>
        </example>

        <example>
            <statement>
                <p>
                    The graph shown is a transformation of the toolkit function <m>f(x)=x^{3}</m>.
                    Relate this new function <m>g(x)</m> to <m>f(x)</m>, then find a formula for <m>g(x)</m>.
                </p>

                <image source="C1S6Image20.png">
                </image>

                <p>
                    When trying to determine a vertical stretch or shift, it is helpful to look for a point on the graph that is relatively clear.
                    In this graph, it appears that <m>g(2)=2</m>.
                    With the basic cubic function at the same input, <m>f(2)=2^{3}=8</m>.
                </p>

                <p>
                    Based on that, it appears that the outputs of <m>g</m> are <m>\dfrac{1}{4}</m> the outputs of the function <m>f</m>, since <m>g(2)=\dfrac{1}{4}f(2)</m>.
                </p>

                <p>
                    From this we can fairly safely conclude that: <m>g(x)=\dfrac{1}{4}f(x)</m>.
                </p>

                <p>
                    We can write a formula for <m>g</m> by using the definition of the function <m>f</m>.
                    <m>g(x)=\dfrac{1}{4}f(x)=\dfrac{1}{4}x^{3}</m>.
                </p>
            </statement>
        </example>

        <p>
            Now we consider changes to the inside of a function.
        </p>

        <example>
            <statement>
                <p>
                    Returning to the fruit fly population we looked at earlier, suppose the scientist is now comparing it to a population that progresses through its lifespan twice as fast as the original population.
                    In other words, this new population, <m>R</m>, will progress in 1 hour the same amount the original population did in 2 hours, and in 2 hours, will progress as much as the original population did in 4 hours.
                    Sketch a graph of this population.
                </p>

                <p>
                    Symbolically, we could write <m>R(1)=P(2)</m>, <m>R(2)=P(4)</m>, and in general, <m>R(t)=P(2t)</m>.
                </p>

                <p>
                    Graphing this,
                </p>

                <image source="C1S6Image18.png">
                </image>
                Original population $P(t)$
                <image source="C1S6Image21.png">
                </image>
                Transformed population $Q(t)$
                <p>
                    Note the effect on the graph is a horizontal compression, where all input values are half their original distance from the vertical axis.
                </p>
            </statement>
        </example>

        <definition>
            <title>Horizontal Stretch/Compression</title>

            <statement>
                <p>
                    Given a function <m>f(x)</m>, if we define a new function <m>g(x)</m> as <m>g(x)=f(kx)</m>, where <m>k</m> is a constant then <m>g(x)</m> is a horizontal stretch or compression of the function <m>f(x)</m>.
                </p>

                <p>
                    If <m>k > 1</m>, then the graph will be compressed by a factor of <m>\dfrac{1}{k}</m>.
                    If <m>0&#x3C; k &#x3C; 1</m>, then the graph will be stretched by a factor of <m>\dfrac{1}{k}</m>.
                    If <m>k &#x3C; 0</m>, then there will be combination of a horizontal stretch or compression with a horizontal reflection.
                </p>
            </statement>
        </definition>

        <example>
            <statement>
                <p>
                    A function <m>f(x)</m> is given as a table below.
                    Create a table for the function <m>g(x)=f(\dfrac{1}{2}x)</m>.
                </p>

                <tabular>
                    <col halign="center"> </col> <col halign="center"> </col> <col halign="center"> </col> <col halign="center"> </col> <col halign="center"> </col>
                    <row>
                        <cell> <m>x</m> </cell>
                        <cell>2</cell>
                        <cell>4</cell>
                        <cell>6</cell>
                        <cell>8</cell>
                    </row>

                    <row>
                        <cell> <m>f(x)</m> </cell>
                        <cell>1</cell>
                        <cell>3</cell>
                        <cell>5</cell>
                        <cell>11</cell>
                    </row>
                </tabular>

                <p>
                    The formula <m>g(x)=f(\dfrac{1}{2}x)</m> tells us that the output values for <m>g</m> are the same as the output values for the function <m>f</m> at an input half the size.
                    Notice that we don’t have enough information to determine <m>g(2)</m> since <m>g(2)=f(\dfrac{1}{2}*2)=f(1)</m>, and we do not have a value for <m>f(1)</m> in our table.
                    Our input values to <m>g</m> will need to be twice as large to get inputs for <m>f</m> that we can evaluate.
                    For example, we can determine <m>g(4)</m> since <m>g(4)=f(\dfrac{1}{2}* 4)=f(2)=1</m>.
                </p>

                <tabular>
                    <col halign="center"> </col> <col halign="center"> </col> <col halign="center"> </col> <col halign="center"> </col> <col halign="center"> </col>
                    <row>
                        <cell> <m>x</m> </cell>
                        <cell>4</cell>
                        <cell>8</cell>
                        <cell>12</cell>
                        <cell>16</cell>
                    </row>

                    <row>
                        <cell> <m>g(x)</m> </cell>
                        <cell>1</cell>
                        <cell>3</cell>
                        <cell>7</cell>
                        <cell>11</cell>
                    </row>
                </tabular>

                <p>
                    Since each input value has been doubled, the result is that the function <m>g(x)</m> has been stretched horizontally by a factor of 2.
                </p>
            </statement>
        </example>

        <example>
            <statement>
                <p>
                    Two graphs are shown below.
                    Relate the function <m>g(x)</m> to <m>f(x)</m>.
                </p>

                <image source="C1S6Image22.png">
                </image>
                Graph of $f(x)$
                <image source="C1S6Image23.png">
                </image>
                Graph of $g(x)$
                <p>
                    The graph of <m>g(x)</m> looks like the graph of <m>f(x)</m> horizontally compressed.
                    Since <m>f(x)</m> ends at <m>(6,4)</m> and <m>g(x)</m> ends at <m>(2,4)</m> we can see that the <m>x</m> values have been compressed by <m>\dfrac{1}{3}</m>, because <m>6(\dfrac{1}{3}) = 2</m>.
                    We might also notice that <m>g(2)=f(6)</m>, and <m>g(1)=f(3)</m>.
                    Either way, we can describe this relationship as <m>g(x)=f(3x)</m>.
                    This is a horizontal compression by <m>\dfrac{1}{3}</m>.
                </p>
            </statement>
        </example>

        <p>
            Notice that the coefficient needed for a horizontal stretch or compression is the reciprocal of the stretch or compression.
            To stretch the graph horizontally by 4, we need a coefficient of <m>\dfrac{1}{4}</m> in our function: <m>f(\dfrac{1}{4}x)</m>.
            This means the input values must be four times larger to produce the same result, requiring the input to be larger, causing the horizontal stretching.
        </p>

        <exercise>
            <statement>
                <p>
                    Write a formula for the toolkit square root function horizontally stretched by three.
                </p>
            </statement>
        </exercise>

        <p>
            It is useful to note that for most toolkit functions, a horizontal stretch or vertical stretch can be represented in other ways.
            For example, a horizontal compression of the function <m>f(x)=x^{2}</m> by <m>\dfrac{1}{2}</m> would result in a new function <m>g(x)=(2x)^{2}</m>, but this can also be written as <m>g(x)=4x^{2}</m>, a vertical stretch of <m>f(x)</m> by 4.
            When writing a formula for a transformed toolkit, we only need to find one transformation that would produce the graph.
        </p>
    </subsection>


    <subsection>
        <title>Combining Transformations</title>

        <p>
            When combining transformations, it is very important to consider the order of the transformations.
            For example, vertically shifting by 3 and then vertically stretching by 2 does not create the same graph as vertically stretching by 2 then vertically shifting by 3.
        </p>

        <p>
            When we see an expression like <m>2f(x)+3</m>, which transformation should we start with? The answer here follows nicely from order of operations, for outside transformations.
            Given the output value of <m>f(x)</m>, we first multiply by 2, causing the vertical stretch, then add 3, causing the vertical shift.
            (Multiplication before Addition)
        </p>

        <definition>
            <title>Combining Vertical Transformations</title>

            <statement>
                <p>
                    When combining vertical transformations written in the form <m>af(x)+k</m>, first vertically stretch by <m>a</m>, then vertically shift by <m>k</m>.
                </p>
            </statement>
        </definition>

        <p>
            Horizontal transformations are a little trickier to think about.
            When we write <m>g(x)=f(2x+3)</m> for example, we have to think about how the inputs to the <m>g</m> function relate to the inputs to the <m>f</m> function.
            Suppose we know <m>f(7)=12</m>.
            What input to <m>g</m> would produce that output? In other words, what value of <m>x</m> will allow <m>g(x)=f(2x+3)=f(12)</m>? We would need <m>2x+3=12</m>.
            To solve for <m>x</m>, we would first subtract 3, resulting in horizontal shift, then divide by 2, causing a horizontal compression.
        </p>

        <definition>
            <title>Combining Horizontal Transformations</title>

            <statement>
                <p>
                    When combining horizontal transformations written in the form <m>f(bx+p)</m>, first horizontally shift by <m>p</m>, then horizontally stretch by <m>\dfrac{1}{b}</m>.
                </p>
            </statement>
        </definition>

        <p>
            This format ends up being very difficult to work with, since it is usually much easier to horizontally stretch a graph before shifting.
            We can work around this by factoring inside the function.
            <me>
                f(bx+p) = f(b(x+\dfrac{p}{b}))
            </me>
            Factoring in this way allows us to horizontally stretch first, then shift horizontally.
        </p>

        <definition>
            <title>Combining Horizontal Transformations (Factored Form)</title>

            <statement>
                <p>
                    When combining horizontal transformations written in the form <m>f(b(x+h))</m>, first horizontally stretch by <m>\dfrac{1}{b}</m>, then horizontally shift by <m>h</m>.
                </p>
            </statement>
        </definition>

        <definition>
            <title>Independence of Horizontal and Vertical Transformations</title>

            <statement>
                <p>
                    Horizontal and vertical transformations are independent.
                    It does not matter whether horizontal or vertical transformations are done first.
                </p>
            </statement>
        </definition>

        <example>
            <statement>
                <p>
                    Given the table of values for the function <m>f(x)</m> below, create a table of values for the function <m>g(x)=2f(3x)+1</m>.
                </p>

                <tabular>
                    <col halign="center"> </col> <col halign="center"> </col> <col halign="center"> </col> <col halign="center"> </col> <col halign="center"> </col>
                    <row>
                        <cell> <m>x</m> </cell>
                        <cell>6</cell>
                        <cell>12</cell>
                        <cell>18</cell>
                        <cell>24</cell>
                    </row>

                    <row>
                        <cell> <m>f(x)</m> </cell>
                        <cell>10</cell>
                        <cell>14</cell>
                        <cell>15</cell>
                        <cell>17</cell>
                    </row>
                </tabular>

                <p>
                    There are 3 steps to this transformation, and we will work from the inside out.
                    Starting with the horizontal transformations, <m>f(3x)</m> is a horizontal compression by <m>\dfrac{1}{3}</m>, which means we multiply each <m>x</m> value by <m>\dfrac{1}{3}</m>.
                </p>

                <tabular>
                    <col halign="center"> </col> <col halign="center"> </col> <col halign="center"> </col> <col halign="center"> </col> <col halign="center"> </col>
                    <row>
                        <cell> <m>x</m> </cell>
                        <cell>2</cell>
                        <cell>4</cell>
                        <cell>6</cell>
                        <cell>8</cell>
                    </row>

                    <row>
                        <cell> <m>f(3x)</m> </cell>
                        <cell>10</cell>
                        <cell>14</cell>
                        <cell>15</cell>
                        <cell>17</cell>
                    </row>
                </tabular>

                <p>
                    Looking now to the vertical transformations, we start with the vertical stretch, which will multiply the output values by 2.
                    We apply this to the previous transformation.
                </p>

                <tabular>
                    <col halign="center"> </col> <col halign="center"> </col> <col halign="center"> </col> <col halign="center"> </col> <col halign="center"> </col>
                    <row>
                        <cell> <m>x</m> </cell>
                        <cell>2</cell>
                        <cell>4</cell>
                        <cell>6</cell>
                        <cell>8</cell>
                    </row>

                    <row>
                        <cell> <m>2f(3x)</m> </cell>
                        <cell>20</cell>
                        <cell>28</cell>
                        <cell>30</cell>
                        <cell>34</cell>
                    </row>
                </tabular>

                <p>
                    Finally, we can apply the vertical shift, which will add 1 to all the output values.
                </p>

                <tabular>
                    <col halign="center"> </col> <col halign="center"> </col> <col halign="center"> </col> <col halign="center"> </col> <col halign="center"> </col>
                    <row>
                        <cell> <m>x</m> </cell>
                        <cell>2</cell>
                        <cell>4</cell>
                        <cell>6</cell>
                        <cell>8</cell>
                    </row>

                    <row>
                        <cell> <m>g(x)=2f(3x)+1</m> </cell>
                        <cell>21</cell>
                        <cell>29</cell>
                        <cell>31</cell>
                        <cell>35</cell>
                    </row>
                </tabular>
            </statement>
        </example>

        <example>
            <statement>
                <p>
                    Using the graph of <m>f(x)</m> below, sketch a graph of <m>k(x)=f(\dfrac{1}{2}x+1)-3</m>.
                </p>

                <image source="C1S6Image24.png">
                </image>

                <p>
                    To make things simpler, we’ll start by factoring out the inside of the function <m>f(\dfrac{1}{2}x+1)-3=f(\dfrac{1}{2}(x+2))-3</m>.
                </p>

                <p>
                    By factoring the inside, we can first horizontally stretch by 2, as indicated by the <m>\dfrac{1}{2}</m> on the inside of the function.
                    Remember twice the size of 0 is still 0, so the point <m>(0,2)</m> remains at <m>(0,2)</m> while the point <m>(2,0)</m> will stretch to <m>(4,0)</m>.
                    Next, we horizontally shift left by 2 units, as indicated by the <m>x+2</m>.
                </p>

                <p>
                    Last, we vertically shift down by 3 to complete our sketch, as indicated by the -3 on the outside of the function.
                </p>

                <image source="C1S6Image25.png">
                </image>
                Horizontal stretch by 2
                <image source="C1S6Image26.png">
                </image>
                Horizontal shift left by 2
                <image source="C1S6Image27.png">
                </image>
                Vertical shift down 3
            </statement>
        </example>

        <example>
            <statement>
                <p>
                    Write an equation for the transformed graph of the quadratic function shown.
                </p>

                <image source="C1S6Image28.png">
                </image>

                <p>
                    Since this is a quadratic function, first consider what the basic quadratic tool kit function looks like and how this has changed.
                    Observing the graph, we notice several transformations:
                </p>

                <p>
                    The original tool kit function has been flipped over the <m>x</m>-axis, some kind of stretch or compression has occurred, and we can see a shift to the right 3 units and a shift up 1 unit.
                </p>

                <p>
                    In total there are four operations:
                    <ul>
                        <li>
                            <p>
                                Vertical reflection, requiring a negative sign outside the function
                            </p>
                        </li>

                        <li>
                            <p>
                                Vertical Stretch or Horizontal Compression*
                            </p>
                        </li>

                        <li>
                            <p>
                                Horizontal Shift Right 3 units, which tells us to put <m>x-3</m> on the inside of the function
                            </p>
                        </li>

                        <li>
                            <p>
                                Vertical Shift up 1 unit, telling us to add 1 on the outside of the function
                            </p>
                        </li>
                    </ul>
                </p>

                <p>
                    * It is unclear from the graph whether it is showing a vertical stretch or a horizontal compression.
                    For the quadratic, it turns out we could represent it either way, so we’ll use a vertical stretch.
                    You may be able to determine the vertical stretch by observation.
                </p>

                <p>
                    By observation, the basic tool kit function has a vertex at <m>(0, 0)</m> and symmetrical points at <m>(1, 1)</m> and <m>(-1, 1)</m>.
                    These points are one unit up and one unit over from the vertex.
                    The new points on the transformed graph are one unit away horizontally but 2 units away vertically.
                    They have been stretched vertically by two.
                </p>

                <p>
                    Not everyone can see this by simply looking at the graph.
                    If you can, great, but if not, we can solve for it.
                    First, we will write the equation for this graph, with an unknown vertical stretch.
                </p>

                <p>
                    The original function is <m>f(x)=x^{2}</m>.
                    To vertically reflect: <m>-f(x)=-x^{2}</m>.
                    To vertically stretch: <m>-af(x)=-ax^{2}</m>.
                    To shift right by 3 units: <m>-af(x-3)=-a(x-3)^{2}</m>.
                    To shift up by 1 unit: <m>-af(x-3)+1=-a(x-3)^{2}+1</m>.
                </p>

                <p>
                    We now know our graph is going to have an equation of the form <m>g(x)=-a(x-3)^{2}+1</m>.
                    To find the vertical stretch, we can identify any point on the graph (other than the highest point), such as the point <m>(2, -1)</m>, which tells us <m>g(2)=-1</m>.
                    Using our general formula, and substituting 2 for <m>x</m>, and -1 for <m>g(x)</m>:
                </p>

                <p>
                    <me>
                        -1=-a(2-3)^{2}+1
                    </me>
                    <me>
                        -1=-a+1
                    </me>
                    <me>
                        -2=-a
                    </me>
                    <me>
                        2=a
                    </me>
                </p>

                <p>
                    This tells us that to produce the graph we need a vertical stretch by two.
                    The function that produces this graph is therefore <m>g(x)=-2(x-3)^{2}+1</m>.
                </p>
            </statement>
        </example>

        <example>
            <statement>
                <p>
                    Consider the linear function <m>g(x)=-2x+1</m>.
                    Describe its transformation in words using the identity tool kit function <m>f(x) = x</m> as a reference.
                </p>
            </statement>
        </example>

        <example>
            <statement>
                <p>
                    On what interval(s) is the function <m>g(x)=\dfrac{-2}{(x-1)^2}+3</m> increasing and decreasing?
                </p>

                <p>
                    This is a transformation of the toolkit reciprocal squared function, <m>f(x)=\dfrac{1}{x^2}</m>:
                </p>

                <p>
                    A vertical flip and vertical stretch by 2: <m>-2f(x)=\dfrac{-2}{x^2}</m>
                </p>

                <p>
                    A shift right by 1: <m>-2f(x-1)=\dfrac{-2}{(x-1)^2}</m>
                </p>

                <p>
                    A shift up by 3: <m>-2f(x-1)+3=\dfrac{-2}{(x-1)^2}+3</m>
                </p>

                <p>
                    The basic reciprocal squared function is increasing on <m>(-\infty,0)</m> and decreasing on <m>(0,\infty)</m>.
                    Because of the vertical flip, the <m>g(x)</m> function will be decreasing on the left and increasing on the right.
                    The horizontal shift right by 1 will also shift these intervals to the right one.
                    From this, we can determine <m>g(x)</m> will be increasing on <m>(1,\infty)</m> and decreasing on <m>(-\infty,1)</m>.
                    We also could graph the transformation to help us determine these intervals.
                </p>

                <image source="C1S6Image29.png">
                </image>
                Graph of $g(x)=\dfrac{-2}{(x-1)^2}+3$.
            </statement>
        </example>

        <example>
            <title>Important Topics of This Section</title>

            <statement>
                <p>
                    <ul>
                        <li>
                            <p>
                                Transformations
                            </p>
                        </li>

                        <li>
                            <p>
                                Vertical Shift (up and down)
                            </p>
                        </li>

                        <li>
                            <p>
                                Horizontal Shifts (left and right)
                            </p>
                        </li>

                        <li>
                            <p>
                                Reflections over the vertical and horizontal axis
                            </p>
                        </li>

                        <li>
                            <p>
                                Vertical Stretches and Compressions
                            </p>
                        </li>

                        <li>
                            <p>
                                Horizontal Stretches and Compressions
                            </p>
                        </li>

                        <li>
                            <p>
                                Combinations of Transformation
                            </p>
                        </li>
                    </ul>
                </p>
            </statement>
        </example>

        <exercise>
            <statement>
                <p>
                    [Exercise Answers]
                    <ol>
                        <li>
                            <p>
                                <m>b(t)=a(t)+10=-4.9t^{2}+30t+10</m>
                            </p>
                        </li>

                        <li>
                            <image source="C1S6Image30.png">
                            </image>
                        </li>

                        <li>
                            <image source="C1S6Image31.png">
                            </image>

                            <p>
                                Notice: g(x) = f(-x) looks the same as f(x)
                            </p>
                        </li>

                        <li>
                            <p>
                                <m>g(x)=f(\dfrac{1}{3}x)</m> so using the square root function we get <m>g(x)=\sqrt{\dfrac{1}{3} x}</m>.
                            </p>
                        </li>

                        <li>
                            <p>
                                The identity tool kit function <m>f(x) = x</m> has been transformed in 3 steps
                                <ol>
                                    <li>
                                        <p>
                                            Vertically stretched by 2.
                                        </p>
                                    </li>

                                    <li>
                                        <p>
                                            Vertically reflected over the <m>x</m>-axis.
                                        </p>
                                    </li>

                                    <li>
                                        <p>
                                            Vertically shifted up by 1 unit.
                                        </p>
                                    </li>
                                </ol>
                            </p>
                        </li>
                    </ol>
                </p>
            </statement>
        </exercise>
    </subsection>
</section>