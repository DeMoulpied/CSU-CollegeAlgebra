<p>
    Chapter one was a window that gave us a peek into the entire course.
    Our goal was to understand the basic structure of functions and function notation, the toolkit functions, domain and range, how to recognize and understand composition and transformations of functions and how to understand and utilize inverse functions.
    With these basic components in hand we will further research the specific details and intricacies of each type of function in our toolkit and use them to model the world around us.
</p>

<definition>
    <title>Mathematical Modeling</title>

    <statement>
        <p>
            As we approach day to day life we often need to quantify the things around us, giving structure and numeric value to various situations.
            This ability to add structure enables us to make choices based on patterns we see that are weighted and systematic.
            With this structure in place we can model and even predict behavior to make decisions.
            Adding a numerical structure to a real world situation is called Mathematical Modeling.
        </p>
    </statement>
</definition>

<p>
    When modeling real world scenarios, there are some common growth patterns that are regularly observed.
    We will devote this chapter and the rest of the book to the study of the functions used to model these growth patterns.
</p>